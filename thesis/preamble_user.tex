% thesis info
\title{複数台LiDARと複数UAVを用いた\\人物追従システムに関する研究}
\author{蒔田 大悟} % please use full-width space in japanese name
\years{令和2年度}
% \years{2021年3月}
\department{大阪市立大学大学院工学研究科}
\fieldandcourse{電子情報系専攻 前期博士課程}
% \department{大阪市立大学工学部}
% \fieldandcourse{電気情報工学科}

\spinetitle{三次元地形上における二足歩行ロボットのモデルベース脚配置計画}
\spineyears{27}
\spinedepartment{大阪市立大学大学院}
\spinefieldandcourse{工学研究科電子情報系専攻}
\spinedepartmentfoot{工 学\\研究科}
% \spinedepartment{大阪市立大学工学部}
% \spinefieldandcourse{工学部電気情報工学科}
% \spinedepartmentfoot{工学部}


%%%%%%%%%%%%%%%%%%%%%%%%%%%%%%%%%%%%%%%%%%%%%%%%%%%%%%%%%%%%%%%%%%%%%%%%%%%%%%%%


% line spacing
\usepackage{setspace}
\doublespacing
% \onehalfspacing % default
% \singlespacing

% useful packages and settings
% table and array
\usepackage{array}
\usepackage{booktabs}   % \toprule, \midrule, \bottomrule in tabbler environment (with good spacing)
\usepackage{longtable}
% \usepackage{tabularx}
% \usepackage{ltablex}
% \renewcommand{\arraystretch}{0.5}
% \renewcommand{\doublerulesep}{1pt}

% caption
\usepackage{subcaption}
\captionsetup{compatibility=false}

% misc
\usepackage[nobreak]{cite}
\usepackage[hyphens]{url}
% \usepackage{siunitx}
% \usepackage{algorithm,algpseudocode}
% \usepackage{minted} % pip install --user pygments

% in­tel­li­gent cross-ref­er­enc­ing
% usage:
% in middle of line: \cref{<label>}, \cref{<label1>,<label2>,...} 
% at head of line:   \Cref{<label>}, \Cref{<label1>,<label2>,...} 
\usepackage[capitalise,noabbrev]{cleveref}
% for japanese
\crefformat{chapter}{#2#1{}章#3}
\crefformat{section}{#2#1{}節#3}
\crefformat{subsection}{#2#1{}節#3}
% \crefname{figure}{図}{図}
% \crefname{table}{表}{表}
\crefname{equation}{式}{式}
\crefname{appendix}{付録}{付録}
\newcommand{\crefrangeconjunction}{--}
\newcommand{\crefpairconjunction}{,}
\newcommand{\crefmiddleconjunction}{,}
\newcommand{\creflastconjunction}{,}
\newcommand{\crefpairgroupconjunction}{,}
\newcommand{\crefmiddlegroupconjunction}{,}
\newcommand{\creflastgroupconjunction}{,}
% for english (only figure and table)
\renewcommand{\figurename}{Fig.~}
\renewcommand{\tablename}{Table~}
\crefname{figure}{Fig.}{Figs.}
\Crefname{figure}{Figure}{Figures}
\crefname{table}{Table}{Tables}

% math font (integral, summation, product)
\ifptex
    \DeclareSymbolFont{cmlargesymbols}{OMX}{cmex}{m}{n}
    % \DeclareMathSymbol{\intop}{\mathop}{cmlargesymbols}{"5A}
    % \def\int{\intop\nolimits}
    % \DeclareMathSymbol{\ointop}{\mathop}{cmlargesymbols}{"49}
    % \def\oint{\ointop\nolimits}
    \DeclareMathSymbol{\sumop}{\mathop}{cmlargesymbols}{"58}
    \let\sum\sumop
    \DeclareMathSymbol{\prodop}{\mathop}{cmlargesymbols}{"59}
    \let\prod\prodop
\fi

% unit
\usepackage{siunitx}
\newcommand{\cm}{\si{\centi\meter}}
\newcommand{\rad}{\si{\radian}}
\newcommand{\degree}{\si{\degree}}
\newcommand{\ms}{\si{\milli\second}}
\newcommand{\degrm}{\mathrm{deg}}
\newcommand{\field}{\mathrm{field}}
\newcommand{\goal}{\mathrm{goal}}
\newcommand{\now}{\mathrm{now}}
\newcommand{\nxt}{\mathrm{next}}
\newcommand{\reachable}{\mathrm{reachable}}
\newcommand{\slope}{\mathrm{slope}}
\newcommand{\init}{\mathrm{init}}


%%%%%%%%%%%%%%%%%%%%%%%%%%%%%%%%%%%%%%%%%%%%%%%%%%%%%%%%%%%%%%%%%%%%%%%%%%%%%%%%


% multi language (incomplete)
\usepackage{ifptex,ifxetex,ifluatex}
\ifluatex
    \usepackage[greek,russian,english,main=japanese]{babel} % for greek, russian, english
    \bxFixCaptionLanguage{default}
    % \usepackage{polyglossia}
    % \setdefaultlanguage{japanese}
    % \setotherlanguages{greek,russian,english}
    % \newfontfamily\greekfont{Liberation Sans}[Mapping=tex-text]
    % \newfontfamily\greekfontsf{Liberation Serif}[Mapping=tex-text]
    % \newfontfamily\cyrillicfont{CMU Serif}[Script=Cyrillic,Mapping=tex-text]
    % \newfontfamily\cyrillicfontsf{CMU Sans Serif}[Script=Cyrillic,Mapping=tex-text]
\else\ifxetex
    \usepackage[greek,russian,english,main=japanese]{babel} % for greek, russian, english
    \bxFixCaptionLanguage{default}
    % \usepackage{polyglossia}
    % \setdefaultlanguage{japanese}
    % \setotherlanguages{greek,russian,english}
    % \newfontfamily\greekfont{Liberation Sans}[Mapping=tex-text]
    % \newfontfamily\greekfontsf{Liberation Serif}[Mapping=tex-text]
    % \newfontfamily\cyrillicfont{CMU Serif}[Script=Cyrillic,Mapping=tex-text]
    % \newfontfamily\cyrillicfontsf{CMU Sans Serif}[Script=Cyrillic,Mapping=tex-text]
\else\ifuptex
    \usepackage[korean,schinese,tchinese,greek,russian,english,japanese]{pxbabel} % for greek, russian, english
\else\ifstrictptex
    \usepackage[korean,schinese,tchinese,greek,russian,english,japanese]{pxbabel} % for greek, russian, english
\fi\fi\fi\fi
